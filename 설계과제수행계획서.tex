\documentclass[12pt,a4paper]{report}
\synctex=1
\usepackage[utf8]{inputenc}
\usepackage[margin=2cm, top=1.5cm, bottom=2cm]{geometry}
\usepackage{graphicx}
\usepackage{libertine}
\usepackage{amsmath}
\usepackage{amssymb}
\usepackage{listings}
\usepackage{pgfornament}
\usepackage{eso-pic}
\usepackage{textcomp}
\usepackage{courier}
\usepackage[hangul]{kotex}

\title{
	\centering
	\pgfornament[width=12cm,color=teal]{84}\\
	\vspace{1cm}
	\fontsize{50}{50} \selectfont {예약 시스템 구축\\설계과제 수행 계획서}\\
		\pgfornament[width=12cm,color=teal]{88}\\
	\vfill}
\author{
	\LARGE
	\begin{tabular}{rl}
		\hline
		교과목명 : & 자료구조와 실습\\
		담당교수 : & 정 준호 교수님\\
		학과 : & 불교학부 \\
		학번 : & 2016110056\\ 
		이름 : & 박승원\\
		날짜 : & \today\\
		\hline
	\end{tabular}\vspace{2cm}
	\\
\includegraphics[width=0.5\textwidth]{logo.jpg}
	}
\date{}


\linespread{1.3}

\begin{document}

\maketitle

%\includegrap

\newpage
\tableofcontents

\noindent
%\chapter{설계과제 개요}
%\section{과제 목표}
%\section{과제 내용}
%\subsection{개발 설계}
%\subsection{프로그램 구현}
%\subsection{마무리}
%\section{과제 수행 방법}
\chapter{설계과제 목표 및 주요 내용}
\section{과제 목표}
소형의 개인 사업자들이 하나의 컴퓨터로 혹은 공유기, 인트라넷을 기반으로 한 내부 망을 갖춘 영업 환경에서 범용으로 사용할 수 있는 예약 프로그램.

일반적으로 대형 업체들은 자신들의 예약 프로그램을 자신의 목적에 특화시켜서 구현한 것으로, 인터넷 예약, 결제 기능까지 갖추어서 사용할 수 있을 것이지만, 개인 사업자들의 경우는 그럴 만한 투자 여력이 없다. 이러한 개인 사업자들이 사용할 수 있는 범용의 예약 프로그램을 만들어 보고자 한다.
\section{과제 내용}
범용의 무료 프로그램이기에 개인사업자들의 각각의 요구에 맞춰 특화된 프로그램을 만들수는 없겠지만, 시설과 예약 시간이라는 두 가지의 추상적인 데이터만으로도 일반적인 모든 경우에 모두 대처할 수 있으리라 생각한다. 

여기에서 시설이라 함은 식당인 경우 각각의 테이블이 하나의 시설이 되고, 숙박업소 등에서는 각각의 방이 하나의 시설이 된다.
\subsection{개발 설계}
\subsubsection{계획서 작성}
C언어의 자료 구조인 리스트를 최대한 활용하여 범용의 예약 프로그램을 만든다.\\
호환성을 고려하여 핵심 부분은 커맨드라인으로 만들고(C언어) \\
유저 인터페이스의 편의를 위하여 따로 frontend를 만든다.(C++, 혹은 자바)

\subsubsection{요구사항 검토}
\begin{itemize}
\item 자신의 업체에 맞게 시설을 설정할 수 있을 것. Txt 파일로 간편하게 설정 가능하게 할 것.
\item 업체 내부에서 인트라넷 환경으로 혹은 공유기로 서버에 있는 예약을 조회 변경할 수 있을 것.
\item 예약 시간 입력이 편리할 것. 

\item 시설별 일, 주, 월별로 예약 상황을 테이블 형식으로 간편하게 볼 수 있게 할 것.
\end{itemize}
\subsubsection{모듈 설계}
소스의 재사용성을 높이기 위해 서버와 클라이언트 를 따로 만들고 이를 바탕으로 프론트 엔드를 추가한다.
\paragraph{Server}
tcpip 통신을 통해 클라이언트와 대화하여 예약 진행.
예약 입력, 예약 삭제, 예약 상황 보기의 기능을 구비한다.
\begin{itemize}
	\item 시간에 따라서 자동으로 정렬되는 삽입함수.
	\item 시설명과 사용 시작 시간을 매개변수로 받아 예약을 삭제하는 함수.
	\item 시설명을 매개 변수로 받아 그 시설의 모든 예약 상황을 보여주는 display 함수.
	\item TcpIp 통신 모듈
\end{itemize}

\paragraph{Client} 
호환성을 위해 콘솔로 인터페이스.
문자열로 서버에 질의하고 문자열의 답문을 보여준다.
\begin{itemize}
	\item 대화형으로 서버와 통신하는 함수.
	\item 커맨드 라인으로 서버와 통신하는 함수.
	\item TcpIp 통신 모듈
\end{itemize}

\paragraph{Front end}
클라이언트 프로그램을 사용하는 프로그램. 그래픽 환경을 조성하여 사용자 편의를 도모함.
클라이언트 프로그램을 사용하여 받은 문자열을 process하여 그래픽 환경으로 보여줌. 예약 입력을 그래픽 환경에서 직관적으로 함.
\begin{itemize}
	\item 클라이언트를 사용하는 사용자 프로그램.
	\item 클라이언트로부터 받은 문자열을 인식하여 테이블 형식으로 나타냄.
	\item 일, 주, 월 단위로 스케일을 변화시켜서 예약상황을 표현할 수 있는 기능.
	\item 테이블의 위치를 클릭하여 예약과 삭제를 쉽게 할 수 있는 기능.
\end{itemize}

\subsection{프로그램 구현}

\paragraph{자료구조}
시설들의 포인터 배열에 예약 시간의 리스트가 물린 2차원적인 리스트를 활용한다.\\
% Graphic for TeX using PGF
% Title: /home/zezeon/Programming/datastructure/reserv.dia
% Creator: Dia v0.97.3
% CreationDate: Fri Oct 14 13:19:25 2016
% For: zezeon
% \usepackage{tikz}
% The following commands are not supported in PSTricks at present
% We define them conditionally, so when they are implemented,
% this pgf file will use them.
\ifx\du\undefined
  \newlength{\du}
\fi
\setlength{\du}{15\unitlength}
\begin{tikzpicture}
\pgftransformxscale{1.000000}
\pgftransformyscale{-1.000000}
\definecolor{dialinecolor}{rgb}{0.000000, 0.000000, 0.000000}
\pgfsetstrokecolor{dialinecolor}
\definecolor{dialinecolor}{rgb}{1.000000, 1.000000, 1.000000}
\pgfsetfillcolor{dialinecolor}
\pgfsetlinewidth{0.100000\du}
\pgfsetdash{}{0pt}
\definecolor{dialinecolor}{rgb}{1.000000, 1.000000, 1.000000}
\pgfsetfillcolor{dialinecolor}
\fill (16.850000\du,12.400000\du)--(16.850000\du,13.800000\du)--(22.760000\du,13.800000\du)--(22.760000\du,12.400000\du)--cycle;
\definecolor{dialinecolor}{rgb}{0.000000, 0.000000, 0.000000}
\pgfsetstrokecolor{dialinecolor}
\draw (16.850000\du,12.400000\du)--(16.850000\du,13.800000\du)--(22.760000\du,13.800000\du)--(22.760000\du,12.400000\du)--cycle;
% setfont left to latex
\definecolor{dialinecolor}{rgb}{0.000000, 0.000000, 0.000000}
\pgfsetstrokecolor{dialinecolor}
\node at (19.805000\du,13.550000\du){Reservation};
\definecolor{dialinecolor}{rgb}{1.000000, 1.000000, 1.000000}
\pgfsetfillcolor{dialinecolor}
\fill (16.850000\du,13.800000\du)--(16.850000\du,18.000000\du)--(22.760000\du,18.000000\du)--(22.760000\du,13.800000\du)--cycle;
\definecolor{dialinecolor}{rgb}{0.000000, 0.000000, 0.000000}
\pgfsetstrokecolor{dialinecolor}
\draw (16.850000\du,13.800000\du)--(16.850000\du,18.000000\du)--(22.760000\du,18.000000\du)--(22.760000\du,13.800000\du)--cycle;
% setfont left to latex
\definecolor{dialinecolor}{rgb}{0.000000, 0.000000, 0.000000}
\pgfsetstrokecolor{dialinecolor}
\node[anchor=west] at (17.000000\du,14.660000\du){+이름};
% setfont left to latex
\definecolor{dialinecolor}{rgb}{0.000000, 0.000000, 0.000000}
\pgfsetstrokecolor{dialinecolor}
\node[anchor=west] at (17.000000\du,15.460000\du){+연락처};
% setfont left to latex
\definecolor{dialinecolor}{rgb}{0.000000, 0.000000, 0.000000}
\pgfsetstrokecolor{dialinecolor}
\node[anchor=west] at (17.000000\du,16.260000\du){+예약시작시간};
% setfont left to latex
\definecolor{dialinecolor}{rgb}{0.000000, 0.000000, 0.000000}
\pgfsetstrokecolor{dialinecolor}
\node[anchor=west] at (17.000000\du,17.060000\du){+예약종료시간};
% setfont left to latex
\definecolor{dialinecolor}{rgb}{0.000000, 0.000000, 0.000000}
\pgfsetstrokecolor{dialinecolor}
\node[anchor=west] at (17.000000\du,17.860000\du){+Node};
\definecolor{dialinecolor}{rgb}{1.000000, 1.000000, 1.000000}
\pgfsetfillcolor{dialinecolor}
\fill (16.850000\du,18.000000\du)--(16.850000\du,18.400000\du)--(22.760000\du,18.400000\du)--(22.760000\du,18.000000\du)--cycle;
\definecolor{dialinecolor}{rgb}{0.000000, 0.000000, 0.000000}
\pgfsetstrokecolor{dialinecolor}
\draw (16.850000\du,18.000000\du)--(16.850000\du,18.400000\du)--(22.760000\du,18.400000\du)--(22.760000\du,18.000000\du)--cycle;
\pgfsetlinewidth{0.100000\du}
\pgfsetdash{}{0pt}
\definecolor{dialinecolor}{rgb}{1.000000, 1.000000, 1.000000}
\pgfsetfillcolor{dialinecolor}
\fill (16.755000\du,36.330000\du)--(16.755000\du,37.730000\du)--(22.665000\du,37.730000\du)--(22.665000\du,36.330000\du)--cycle;
\definecolor{dialinecolor}{rgb}{0.000000, 0.000000, 0.000000}
\pgfsetstrokecolor{dialinecolor}
\draw (16.755000\du,36.330000\du)--(16.755000\du,37.730000\du)--(22.665000\du,37.730000\du)--(22.665000\du,36.330000\du)--cycle;
% setfont left to latex
\definecolor{dialinecolor}{rgb}{0.000000, 0.000000, 0.000000}
\pgfsetstrokecolor{dialinecolor}
\node at (19.710000\du,37.480000\du){Reservation};
\definecolor{dialinecolor}{rgb}{1.000000, 1.000000, 1.000000}
\pgfsetfillcolor{dialinecolor}
\fill (16.755000\du,37.730000\du)--(16.755000\du,41.930000\du)--(22.665000\du,41.930000\du)--(22.665000\du,37.730000\du)--cycle;
\definecolor{dialinecolor}{rgb}{0.000000, 0.000000, 0.000000}
\pgfsetstrokecolor{dialinecolor}
\draw (16.755000\du,37.730000\du)--(16.755000\du,41.930000\du)--(22.665000\du,41.930000\du)--(22.665000\du,37.730000\du)--cycle;
% setfont left to latex
\definecolor{dialinecolor}{rgb}{0.000000, 0.000000, 0.000000}
\pgfsetstrokecolor{dialinecolor}
\node[anchor=west] at (16.905000\du,38.590000\du){+이름};
% setfont left to latex
\definecolor{dialinecolor}{rgb}{0.000000, 0.000000, 0.000000}
\pgfsetstrokecolor{dialinecolor}
\node[anchor=west] at (16.905000\du,39.390000\du){+연락처};
% setfont left to latex
\definecolor{dialinecolor}{rgb}{0.000000, 0.000000, 0.000000}
\pgfsetstrokecolor{dialinecolor}
\node[anchor=west] at (16.905000\du,40.190000\du){+예약시작시간};
% setfont left to latex
\definecolor{dialinecolor}{rgb}{0.000000, 0.000000, 0.000000}
\pgfsetstrokecolor{dialinecolor}
\node[anchor=west] at (16.905000\du,40.990000\du){+예약종료시간};
% setfont left to latex
\definecolor{dialinecolor}{rgb}{0.000000, 0.000000, 0.000000}
\pgfsetstrokecolor{dialinecolor}
\node[anchor=west] at (16.905000\du,41.790000\du){+Node};
\definecolor{dialinecolor}{rgb}{1.000000, 1.000000, 1.000000}
\pgfsetfillcolor{dialinecolor}
\fill (16.755000\du,41.930000\du)--(16.755000\du,42.330000\du)--(22.665000\du,42.330000\du)--(22.665000\du,41.930000\du)--cycle;
\definecolor{dialinecolor}{rgb}{0.000000, 0.000000, 0.000000}
\pgfsetstrokecolor{dialinecolor}
\draw (16.755000\du,41.930000\du)--(16.755000\du,42.330000\du)--(22.665000\du,42.330000\du)--(22.665000\du,41.930000\du)--cycle;
\pgfsetlinewidth{0.100000\du}
\pgfsetdash{}{0pt}
\definecolor{dialinecolor}{rgb}{1.000000, 1.000000, 1.000000}
\pgfsetfillcolor{dialinecolor}
\fill (16.670000\du,28.060000\du)--(16.670000\du,29.460000\du)--(22.580000\du,29.460000\du)--(22.580000\du,28.060000\du)--cycle;
\definecolor{dialinecolor}{rgb}{0.000000, 0.000000, 0.000000}
\pgfsetstrokecolor{dialinecolor}
\draw (16.670000\du,28.060000\du)--(16.670000\du,29.460000\du)--(22.580000\du,29.460000\du)--(22.580000\du,28.060000\du)--cycle;
% setfont left to latex
\definecolor{dialinecolor}{rgb}{0.000000, 0.000000, 0.000000}
\pgfsetstrokecolor{dialinecolor}
\node at (19.625000\du,29.210000\du){Reservation};
\definecolor{dialinecolor}{rgb}{1.000000, 1.000000, 1.000000}
\pgfsetfillcolor{dialinecolor}
\fill (16.670000\du,29.460000\du)--(16.670000\du,33.660000\du)--(22.580000\du,33.660000\du)--(22.580000\du,29.460000\du)--cycle;
\definecolor{dialinecolor}{rgb}{0.000000, 0.000000, 0.000000}
\pgfsetstrokecolor{dialinecolor}
\draw (16.670000\du,29.460000\du)--(16.670000\du,33.660000\du)--(22.580000\du,33.660000\du)--(22.580000\du,29.460000\du)--cycle;
% setfont left to latex
\definecolor{dialinecolor}{rgb}{0.000000, 0.000000, 0.000000}
\pgfsetstrokecolor{dialinecolor}
\node[anchor=west] at (16.820000\du,30.320000\du){+이름};
% setfont left to latex
\definecolor{dialinecolor}{rgb}{0.000000, 0.000000, 0.000000}
\pgfsetstrokecolor{dialinecolor}
\node[anchor=west] at (16.820000\du,31.120000\du){+연락처};
% setfont left to latex
\definecolor{dialinecolor}{rgb}{0.000000, 0.000000, 0.000000}
\pgfsetstrokecolor{dialinecolor}
\node[anchor=west] at (16.820000\du,31.920000\du){+예약시작시간};
% setfont left to latex
\definecolor{dialinecolor}{rgb}{0.000000, 0.000000, 0.000000}
\pgfsetstrokecolor{dialinecolor}
\node[anchor=west] at (16.820000\du,32.720000\du){+예약종료시간};
% setfont left to latex
\definecolor{dialinecolor}{rgb}{0.000000, 0.000000, 0.000000}
\pgfsetstrokecolor{dialinecolor}
\node[anchor=west] at (16.820000\du,33.520000\du){+Node};
\definecolor{dialinecolor}{rgb}{1.000000, 1.000000, 1.000000}
\pgfsetfillcolor{dialinecolor}
\fill (16.670000\du,33.660000\du)--(16.670000\du,34.060000\du)--(22.580000\du,34.060000\du)--(22.580000\du,33.660000\du)--cycle;
\definecolor{dialinecolor}{rgb}{0.000000, 0.000000, 0.000000}
\pgfsetstrokecolor{dialinecolor}
\draw (16.670000\du,33.660000\du)--(16.670000\du,34.060000\du)--(22.580000\du,34.060000\du)--(22.580000\du,33.660000\du)--cycle;
\pgfsetlinewidth{0.100000\du}
\pgfsetdash{}{0pt}
\definecolor{dialinecolor}{rgb}{1.000000, 1.000000, 1.000000}
\pgfsetfillcolor{dialinecolor}
\fill (16.785000\du,20.390000\du)--(16.785000\du,21.790000\du)--(22.695000\du,21.790000\du)--(22.695000\du,20.390000\du)--cycle;
\definecolor{dialinecolor}{rgb}{0.000000, 0.000000, 0.000000}
\pgfsetstrokecolor{dialinecolor}
\draw (16.785000\du,20.390000\du)--(16.785000\du,21.790000\du)--(22.695000\du,21.790000\du)--(22.695000\du,20.390000\du)--cycle;
% setfont left to latex
\definecolor{dialinecolor}{rgb}{0.000000, 0.000000, 0.000000}
\pgfsetstrokecolor{dialinecolor}
\node at (19.740000\du,21.540000\du){Reservation};
\definecolor{dialinecolor}{rgb}{1.000000, 1.000000, 1.000000}
\pgfsetfillcolor{dialinecolor}
\fill (16.785000\du,21.790000\du)--(16.785000\du,25.990000\du)--(22.695000\du,25.990000\du)--(22.695000\du,21.790000\du)--cycle;
\definecolor{dialinecolor}{rgb}{0.000000, 0.000000, 0.000000}
\pgfsetstrokecolor{dialinecolor}
\draw (16.785000\du,21.790000\du)--(16.785000\du,25.990000\du)--(22.695000\du,25.990000\du)--(22.695000\du,21.790000\du)--cycle;
% setfont left to latex
\definecolor{dialinecolor}{rgb}{0.000000, 0.000000, 0.000000}
\pgfsetstrokecolor{dialinecolor}
\node[anchor=west] at (16.935000\du,22.650000\du){+이름};
% setfont left to latex
\definecolor{dialinecolor}{rgb}{0.000000, 0.000000, 0.000000}
\pgfsetstrokecolor{dialinecolor}
\node[anchor=west] at (16.935000\du,23.450000\du){+연락처};
% setfont left to latex
\definecolor{dialinecolor}{rgb}{0.000000, 0.000000, 0.000000}
\pgfsetstrokecolor{dialinecolor}
\node[anchor=west] at (16.935000\du,24.250000\du){+예약시작시간};
% setfont left to latex
\definecolor{dialinecolor}{rgb}{0.000000, 0.000000, 0.000000}
\pgfsetstrokecolor{dialinecolor}
\node[anchor=west] at (16.935000\du,25.050000\du){+예약종료시간};
% setfont left to latex
\definecolor{dialinecolor}{rgb}{0.000000, 0.000000, 0.000000}
\pgfsetstrokecolor{dialinecolor}
\node[anchor=west] at (16.935000\du,25.850000\du){+Node};
\definecolor{dialinecolor}{rgb}{1.000000, 1.000000, 1.000000}
\pgfsetfillcolor{dialinecolor}
\fill (16.785000\du,25.990000\du)--(16.785000\du,26.390000\du)--(22.695000\du,26.390000\du)--(22.695000\du,25.990000\du)--cycle;
\definecolor{dialinecolor}{rgb}{0.000000, 0.000000, 0.000000}
\pgfsetstrokecolor{dialinecolor}
\draw (16.785000\du,25.990000\du)--(16.785000\du,26.390000\du)--(22.695000\du,26.390000\du)--(22.695000\du,25.990000\du)--cycle;
\pgfsetlinewidth{0.100000\du}
\pgfsetbuttcap
\pgfsetdash{}{0pt}
{
\definecolor{dialinecolor}{rgb}{0.000000, 0.000000, 0.000000}
\pgfsetfillcolor{dialinecolor}
% was here!!!
\pgfsetarrowsstart{latex}
\definecolor{dialinecolor}{rgb}{0.000000, 0.000000, 0.000000}
\pgfsetstrokecolor{dialinecolor}
\draw (19.764788\du,20.343032\du)--(19.780212\du,18.446968\du);
}
% setfont left to latex
\pgfsetlinewidth{0.100000\du}
\pgfsetbuttcap
\pgfsetdash{}{0pt}
{
\definecolor{dialinecolor}{rgb}{0.000000, 0.000000, 0.000000}
\pgfsetfillcolor{dialinecolor}
% was here!!!
\pgfsetarrowsstart{latex}
\definecolor{dialinecolor}{rgb}{0.000000, 0.000000, 0.000000}
\pgfsetstrokecolor{dialinecolor}
\draw (19.670722\du,28.010538\du)--(19.694278\du,26.439462\du);
}
% setfont left to latex
\pgfsetlinewidth{0.100000\du}
\pgfsetbuttcap
\pgfsetdash{}{0pt}
{
\definecolor{dialinecolor}{rgb}{0.000000, 0.000000, 0.000000}
\pgfsetfillcolor{dialinecolor}
% was here!!!
\pgfsetarrowsstart{latex}
\definecolor{dialinecolor}{rgb}{0.000000, 0.000000, 0.000000}
\pgfsetstrokecolor{dialinecolor}
\draw (19.678654\du,36.280236\du)--(19.656346\du,34.109764\du);
}
% setfont left to latex
\pgfsetlinewidth{0.100000\du}
\pgfsetdash{}{0pt}
\definecolor{dialinecolor}{rgb}{1.000000, 1.000000, 1.000000}
\pgfsetfillcolor{dialinecolor}
\fill (26.175000\du,12.350000\du)--(26.175000\du,13.750000\du)--(32.085000\du,13.750000\du)--(32.085000\du,12.350000\du)--cycle;
\definecolor{dialinecolor}{rgb}{0.000000, 0.000000, 0.000000}
\pgfsetstrokecolor{dialinecolor}
\draw (26.175000\du,12.350000\du)--(26.175000\du,13.750000\du)--(32.085000\du,13.750000\du)--(32.085000\du,12.350000\du)--cycle;
% setfont left to latex
\definecolor{dialinecolor}{rgb}{0.000000, 0.000000, 0.000000}
\pgfsetstrokecolor{dialinecolor}
\node at (29.130000\du,13.500000\du){Reservation};
\definecolor{dialinecolor}{rgb}{1.000000, 1.000000, 1.000000}
\pgfsetfillcolor{dialinecolor}
\fill (26.175000\du,13.750000\du)--(26.175000\du,17.950000\du)--(32.085000\du,17.950000\du)--(32.085000\du,13.750000\du)--cycle;
\definecolor{dialinecolor}{rgb}{0.000000, 0.000000, 0.000000}
\pgfsetstrokecolor{dialinecolor}
\draw (26.175000\du,13.750000\du)--(26.175000\du,17.950000\du)--(32.085000\du,17.950000\du)--(32.085000\du,13.750000\du)--cycle;
% setfont left to latex
\definecolor{dialinecolor}{rgb}{0.000000, 0.000000, 0.000000}
\pgfsetstrokecolor{dialinecolor}
\node[anchor=west] at (26.325000\du,14.610000\du){+이름};
% setfont left to latex
\definecolor{dialinecolor}{rgb}{0.000000, 0.000000, 0.000000}
\pgfsetstrokecolor{dialinecolor}
\node[anchor=west] at (26.325000\du,15.410000\du){+연락처};
% setfont left to latex
\definecolor{dialinecolor}{rgb}{0.000000, 0.000000, 0.000000}
\pgfsetstrokecolor{dialinecolor}
\node[anchor=west] at (26.325000\du,16.210000\du){+예약시작시간};
% setfont left to latex
\definecolor{dialinecolor}{rgb}{0.000000, 0.000000, 0.000000}
\pgfsetstrokecolor{dialinecolor}
\node[anchor=west] at (26.325000\du,17.010000\du){+예약종료시간};
% setfont left to latex
\definecolor{dialinecolor}{rgb}{0.000000, 0.000000, 0.000000}
\pgfsetstrokecolor{dialinecolor}
\node[anchor=west] at (26.325000\du,17.810000\du){+Node};
\definecolor{dialinecolor}{rgb}{1.000000, 1.000000, 1.000000}
\pgfsetfillcolor{dialinecolor}
\fill (26.175000\du,17.950000\du)--(26.175000\du,18.350000\du)--(32.085000\du,18.350000\du)--(32.085000\du,17.950000\du)--cycle;
\definecolor{dialinecolor}{rgb}{0.000000, 0.000000, 0.000000}
\pgfsetstrokecolor{dialinecolor}
\draw (26.175000\du,17.950000\du)--(26.175000\du,18.350000\du)--(32.085000\du,18.350000\du)--(32.085000\du,17.950000\du)--cycle;
\pgfsetlinewidth{0.100000\du}
\pgfsetdash{}{0pt}
\definecolor{dialinecolor}{rgb}{1.000000, 1.000000, 1.000000}
\pgfsetfillcolor{dialinecolor}
\fill (26.230000\du,36.680000\du)--(26.230000\du,38.080000\du)--(32.140000\du,38.080000\du)--(32.140000\du,36.680000\du)--cycle;
\definecolor{dialinecolor}{rgb}{0.000000, 0.000000, 0.000000}
\pgfsetstrokecolor{dialinecolor}
\draw (26.230000\du,36.680000\du)--(26.230000\du,38.080000\du)--(32.140000\du,38.080000\du)--(32.140000\du,36.680000\du)--cycle;
% setfont left to latex
\definecolor{dialinecolor}{rgb}{0.000000, 0.000000, 0.000000}
\pgfsetstrokecolor{dialinecolor}
\node at (29.185000\du,37.830000\du){Reservation};
\definecolor{dialinecolor}{rgb}{1.000000, 1.000000, 1.000000}
\pgfsetfillcolor{dialinecolor}
\fill (26.230000\du,38.080000\du)--(26.230000\du,42.280000\du)--(32.140000\du,42.280000\du)--(32.140000\du,38.080000\du)--cycle;
\definecolor{dialinecolor}{rgb}{0.000000, 0.000000, 0.000000}
\pgfsetstrokecolor{dialinecolor}
\draw (26.230000\du,38.080000\du)--(26.230000\du,42.280000\du)--(32.140000\du,42.280000\du)--(32.140000\du,38.080000\du)--cycle;
% setfont left to latex
\definecolor{dialinecolor}{rgb}{0.000000, 0.000000, 0.000000}
\pgfsetstrokecolor{dialinecolor}
\node[anchor=west] at (26.380000\du,38.940000\du){+이름};
% setfont left to latex
\definecolor{dialinecolor}{rgb}{0.000000, 0.000000, 0.000000}
\pgfsetstrokecolor{dialinecolor}
\node[anchor=west] at (26.380000\du,39.740000\du){+연락처};
% setfont left to latex
\definecolor{dialinecolor}{rgb}{0.000000, 0.000000, 0.000000}
\pgfsetstrokecolor{dialinecolor}
\node[anchor=west] at (26.380000\du,40.540000\du){+예약시작시간};
% setfont left to latex
\definecolor{dialinecolor}{rgb}{0.000000, 0.000000, 0.000000}
\pgfsetstrokecolor{dialinecolor}
\node[anchor=west] at (26.380000\du,41.340000\du){+예약종료시간};
% setfont left to latex
\definecolor{dialinecolor}{rgb}{0.000000, 0.000000, 0.000000}
\pgfsetstrokecolor{dialinecolor}
\node[anchor=west] at (26.380000\du,42.140000\du){+Node};
\definecolor{dialinecolor}{rgb}{1.000000, 1.000000, 1.000000}
\pgfsetfillcolor{dialinecolor}
\fill (26.230000\du,42.280000\du)--(26.230000\du,42.680000\du)--(32.140000\du,42.680000\du)--(32.140000\du,42.280000\du)--cycle;
\definecolor{dialinecolor}{rgb}{0.000000, 0.000000, 0.000000}
\pgfsetstrokecolor{dialinecolor}
\draw (26.230000\du,42.280000\du)--(26.230000\du,42.680000\du)--(32.140000\du,42.680000\du)--(32.140000\du,42.280000\du)--cycle;
\pgfsetlinewidth{0.100000\du}
\pgfsetdash{}{0pt}
\definecolor{dialinecolor}{rgb}{1.000000, 1.000000, 1.000000}
\pgfsetfillcolor{dialinecolor}
\fill (26.195000\du,28.460000\du)--(26.195000\du,29.860000\du)--(32.105000\du,29.860000\du)--(32.105000\du,28.460000\du)--cycle;
\definecolor{dialinecolor}{rgb}{0.000000, 0.000000, 0.000000}
\pgfsetstrokecolor{dialinecolor}
\draw (26.195000\du,28.460000\du)--(26.195000\du,29.860000\du)--(32.105000\du,29.860000\du)--(32.105000\du,28.460000\du)--cycle;
% setfont left to latex
\definecolor{dialinecolor}{rgb}{0.000000, 0.000000, 0.000000}
\pgfsetstrokecolor{dialinecolor}
\node at (29.150000\du,29.610000\du){Reservation};
\definecolor{dialinecolor}{rgb}{1.000000, 1.000000, 1.000000}
\pgfsetfillcolor{dialinecolor}
\fill (26.195000\du,29.860000\du)--(26.195000\du,34.060000\du)--(32.105000\du,34.060000\du)--(32.105000\du,29.860000\du)--cycle;
\definecolor{dialinecolor}{rgb}{0.000000, 0.000000, 0.000000}
\pgfsetstrokecolor{dialinecolor}
\draw (26.195000\du,29.860000\du)--(26.195000\du,34.060000\du)--(32.105000\du,34.060000\du)--(32.105000\du,29.860000\du)--cycle;
% setfont left to latex
\definecolor{dialinecolor}{rgb}{0.000000, 0.000000, 0.000000}
\pgfsetstrokecolor{dialinecolor}
\node[anchor=west] at (26.345000\du,30.720000\du){+이름};
% setfont left to latex
\definecolor{dialinecolor}{rgb}{0.000000, 0.000000, 0.000000}
\pgfsetstrokecolor{dialinecolor}
\node[anchor=west] at (26.345000\du,31.520000\du){+연락처};
% setfont left to latex
\definecolor{dialinecolor}{rgb}{0.000000, 0.000000, 0.000000}
\pgfsetstrokecolor{dialinecolor}
\node[anchor=west] at (26.345000\du,32.320000\du){+예약시작시간};
% setfont left to latex
\definecolor{dialinecolor}{rgb}{0.000000, 0.000000, 0.000000}
\pgfsetstrokecolor{dialinecolor}
\node[anchor=west] at (26.345000\du,33.120000\du){+예약종료시간};
% setfont left to latex
\definecolor{dialinecolor}{rgb}{0.000000, 0.000000, 0.000000}
\pgfsetstrokecolor{dialinecolor}
\node[anchor=west] at (26.345000\du,33.920000\du){+Node};
\definecolor{dialinecolor}{rgb}{1.000000, 1.000000, 1.000000}
\pgfsetfillcolor{dialinecolor}
\fill (26.195000\du,34.060000\du)--(26.195000\du,34.460000\du)--(32.105000\du,34.460000\du)--(32.105000\du,34.060000\du)--cycle;
\definecolor{dialinecolor}{rgb}{0.000000, 0.000000, 0.000000}
\pgfsetstrokecolor{dialinecolor}
\draw (26.195000\du,34.060000\du)--(26.195000\du,34.460000\du)--(32.105000\du,34.460000\du)--(32.105000\du,34.060000\du)--cycle;
\pgfsetlinewidth{0.100000\du}
\pgfsetdash{}{0pt}
\definecolor{dialinecolor}{rgb}{1.000000, 1.000000, 1.000000}
\pgfsetfillcolor{dialinecolor}
\fill (26.110000\du,20.540000\du)--(26.110000\du,21.940000\du)--(32.020000\du,21.940000\du)--(32.020000\du,20.540000\du)--cycle;
\definecolor{dialinecolor}{rgb}{0.000000, 0.000000, 0.000000}
\pgfsetstrokecolor{dialinecolor}
\draw (26.110000\du,20.540000\du)--(26.110000\du,21.940000\du)--(32.020000\du,21.940000\du)--(32.020000\du,20.540000\du)--cycle;
% setfont left to latex
\definecolor{dialinecolor}{rgb}{0.000000, 0.000000, 0.000000}
\pgfsetstrokecolor{dialinecolor}
\node at (29.065000\du,21.690000\du){Reservation};
\definecolor{dialinecolor}{rgb}{1.000000, 1.000000, 1.000000}
\pgfsetfillcolor{dialinecolor}
\fill (26.110000\du,21.940000\du)--(26.110000\du,26.140000\du)--(32.020000\du,26.140000\du)--(32.020000\du,21.940000\du)--cycle;
\definecolor{dialinecolor}{rgb}{0.000000, 0.000000, 0.000000}
\pgfsetstrokecolor{dialinecolor}
\draw (26.110000\du,21.940000\du)--(26.110000\du,26.140000\du)--(32.020000\du,26.140000\du)--(32.020000\du,21.940000\du)--cycle;
% setfont left to latex
\definecolor{dialinecolor}{rgb}{0.000000, 0.000000, 0.000000}
\pgfsetstrokecolor{dialinecolor}
\node[anchor=west] at (26.260000\du,22.800000\du){+이름};
% setfont left to latex
\definecolor{dialinecolor}{rgb}{0.000000, 0.000000, 0.000000}
\pgfsetstrokecolor{dialinecolor}
\node[anchor=west] at (26.260000\du,23.600000\du){+연락처};
% setfont left to latex
\definecolor{dialinecolor}{rgb}{0.000000, 0.000000, 0.000000}
\pgfsetstrokecolor{dialinecolor}
\node[anchor=west] at (26.260000\du,24.400000\du){+예약시작시간};
% setfont left to latex
\definecolor{dialinecolor}{rgb}{0.000000, 0.000000, 0.000000}
\pgfsetstrokecolor{dialinecolor}
\node[anchor=west] at (26.260000\du,25.200000\du){+예약종료시간};
% setfont left to latex
\definecolor{dialinecolor}{rgb}{0.000000, 0.000000, 0.000000}
\pgfsetstrokecolor{dialinecolor}
\node[anchor=west] at (26.260000\du,26.000000\du){+Node};
\definecolor{dialinecolor}{rgb}{1.000000, 1.000000, 1.000000}
\pgfsetfillcolor{dialinecolor}
\fill (26.110000\du,26.140000\du)--(26.110000\du,26.540000\du)--(32.020000\du,26.540000\du)--(32.020000\du,26.140000\du)--cycle;
\definecolor{dialinecolor}{rgb}{0.000000, 0.000000, 0.000000}
\pgfsetstrokecolor{dialinecolor}
\draw (26.110000\du,26.140000\du)--(26.110000\du,26.540000\du)--(32.020000\du,26.540000\du)--(32.020000\du,26.140000\du)--cycle;
\pgfsetlinewidth{0.100000\du}
\pgfsetbuttcap
\pgfsetdash{}{0pt}
{
\definecolor{dialinecolor}{rgb}{0.000000, 0.000000, 0.000000}
\pgfsetfillcolor{dialinecolor}
% was here!!!
\pgfsetarrowsstart{latex}
\definecolor{dialinecolor}{rgb}{0.000000, 0.000000, 0.000000}
\pgfsetstrokecolor{dialinecolor}
\draw (29.089204\du,20.490245\du)--(29.105796\du,18.399755\du);
}
% setfont left to latex
\pgfsetlinewidth{0.100000\du}
\pgfsetbuttcap
\pgfsetdash{}{0pt}
{
\definecolor{dialinecolor}{rgb}{0.000000, 0.000000, 0.000000}
\pgfsetfillcolor{dialinecolor}
% was here!!!
\pgfsetarrowsstart{latex}
\definecolor{dialinecolor}{rgb}{0.000000, 0.000000, 0.000000}
\pgfsetstrokecolor{dialinecolor}
\draw (29.117269\du,28.410239\du)--(29.097731\du,26.589761\du);
}
% setfont left to latex
\pgfsetlinewidth{0.100000\du}
\pgfsetbuttcap
\pgfsetdash{}{0pt}
{
\definecolor{dialinecolor}{rgb}{0.000000, 0.000000, 0.000000}
\pgfsetfillcolor{dialinecolor}
% was here!!!
\pgfsetarrowsstart{latex}
\definecolor{dialinecolor}{rgb}{0.000000, 0.000000, 0.000000}
\pgfsetstrokecolor{dialinecolor}
\draw (29.172080\du,36.645664\du)--(29.162920\du,34.494336\du);
}
% setfont left to latex
\pgfsetlinewidth{0.100000\du}
\pgfsetdash{}{0pt}
\definecolor{dialinecolor}{rgb}{1.000000, 1.000000, 1.000000}
\pgfsetfillcolor{dialinecolor}
\fill (34.840000\du,12.480000\du)--(34.840000\du,13.880000\du)--(40.750000\du,13.880000\du)--(40.750000\du,12.480000\du)--cycle;
\definecolor{dialinecolor}{rgb}{0.000000, 0.000000, 0.000000}
\pgfsetstrokecolor{dialinecolor}
\draw (34.840000\du,12.480000\du)--(34.840000\du,13.880000\du)--(40.750000\du,13.880000\du)--(40.750000\du,12.480000\du)--cycle;
% setfont left to latex
\definecolor{dialinecolor}{rgb}{0.000000, 0.000000, 0.000000}
\pgfsetstrokecolor{dialinecolor}
\node at (37.795000\du,13.630000\du){Reservation};
\definecolor{dialinecolor}{rgb}{1.000000, 1.000000, 1.000000}
\pgfsetfillcolor{dialinecolor}
\fill (34.840000\du,13.880000\du)--(34.840000\du,18.080000\du)--(40.750000\du,18.080000\du)--(40.750000\du,13.880000\du)--cycle;
\definecolor{dialinecolor}{rgb}{0.000000, 0.000000, 0.000000}
\pgfsetstrokecolor{dialinecolor}
\draw (34.840000\du,13.880000\du)--(34.840000\du,18.080000\du)--(40.750000\du,18.080000\du)--(40.750000\du,13.880000\du)--cycle;
% setfont left to latex
\definecolor{dialinecolor}{rgb}{0.000000, 0.000000, 0.000000}
\pgfsetstrokecolor{dialinecolor}
\node[anchor=west] at (34.990000\du,14.740000\du){+이름};
% setfont left to latex
\definecolor{dialinecolor}{rgb}{0.000000, 0.000000, 0.000000}
\pgfsetstrokecolor{dialinecolor}
\node[anchor=west] at (34.990000\du,15.540000\du){+연락처};
% setfont left to latex
\definecolor{dialinecolor}{rgb}{0.000000, 0.000000, 0.000000}
\pgfsetstrokecolor{dialinecolor}
\node[anchor=west] at (34.990000\du,16.340000\du){+예약시작시간};
% setfont left to latex
\definecolor{dialinecolor}{rgb}{0.000000, 0.000000, 0.000000}
\pgfsetstrokecolor{dialinecolor}
\node[anchor=west] at (34.990000\du,17.140000\du){+예약종료시간};
% setfont left to latex
\definecolor{dialinecolor}{rgb}{0.000000, 0.000000, 0.000000}
\pgfsetstrokecolor{dialinecolor}
\node[anchor=west] at (34.990000\du,17.940000\du){+Node};
\definecolor{dialinecolor}{rgb}{1.000000, 1.000000, 1.000000}
\pgfsetfillcolor{dialinecolor}
\fill (34.840000\du,18.080000\du)--(34.840000\du,18.480000\du)--(40.750000\du,18.480000\du)--(40.750000\du,18.080000\du)--cycle;
\definecolor{dialinecolor}{rgb}{0.000000, 0.000000, 0.000000}
\pgfsetstrokecolor{dialinecolor}
\draw (34.840000\du,18.080000\du)--(34.840000\du,18.480000\du)--(40.750000\du,18.480000\du)--(40.750000\du,18.080000\du)--cycle;
\pgfsetlinewidth{0.100000\du}
\pgfsetdash{}{0pt}
\definecolor{dialinecolor}{rgb}{1.000000, 1.000000, 1.000000}
\pgfsetfillcolor{dialinecolor}
\fill (34.945000\du,36.810000\du)--(34.945000\du,38.210000\du)--(40.855000\du,38.210000\du)--(40.855000\du,36.810000\du)--cycle;
\definecolor{dialinecolor}{rgb}{0.000000, 0.000000, 0.000000}
\pgfsetstrokecolor{dialinecolor}
\draw (34.945000\du,36.810000\du)--(34.945000\du,38.210000\du)--(40.855000\du,38.210000\du)--(40.855000\du,36.810000\du)--cycle;
% setfont left to latex
\definecolor{dialinecolor}{rgb}{0.000000, 0.000000, 0.000000}
\pgfsetstrokecolor{dialinecolor}
\node at (37.900000\du,37.960000\du){Reservation};
\definecolor{dialinecolor}{rgb}{1.000000, 1.000000, 1.000000}
\pgfsetfillcolor{dialinecolor}
\fill (34.945000\du,38.210000\du)--(34.945000\du,42.410000\du)--(40.855000\du,42.410000\du)--(40.855000\du,38.210000\du)--cycle;
\definecolor{dialinecolor}{rgb}{0.000000, 0.000000, 0.000000}
\pgfsetstrokecolor{dialinecolor}
\draw (34.945000\du,38.210000\du)--(34.945000\du,42.410000\du)--(40.855000\du,42.410000\du)--(40.855000\du,38.210000\du)--cycle;
% setfont left to latex
\definecolor{dialinecolor}{rgb}{0.000000, 0.000000, 0.000000}
\pgfsetstrokecolor{dialinecolor}
\node[anchor=west] at (35.095000\du,39.070000\du){+이름};
% setfont left to latex
\definecolor{dialinecolor}{rgb}{0.000000, 0.000000, 0.000000}
\pgfsetstrokecolor{dialinecolor}
\node[anchor=west] at (35.095000\du,39.870000\du){+연락처};
% setfont left to latex
\definecolor{dialinecolor}{rgb}{0.000000, 0.000000, 0.000000}
\pgfsetstrokecolor{dialinecolor}
\node[anchor=west] at (35.095000\du,40.670000\du){+예약시작시간};
% setfont left to latex
\definecolor{dialinecolor}{rgb}{0.000000, 0.000000, 0.000000}
\pgfsetstrokecolor{dialinecolor}
\node[anchor=west] at (35.095000\du,41.470000\du){+예약종료시간};
% setfont left to latex
\definecolor{dialinecolor}{rgb}{0.000000, 0.000000, 0.000000}
\pgfsetstrokecolor{dialinecolor}
\node[anchor=west] at (35.095000\du,42.270000\du){+Node};
\definecolor{dialinecolor}{rgb}{1.000000, 1.000000, 1.000000}
\pgfsetfillcolor{dialinecolor}
\fill (34.945000\du,42.410000\du)--(34.945000\du,42.810000\du)--(40.855000\du,42.810000\du)--(40.855000\du,42.410000\du)--cycle;
\definecolor{dialinecolor}{rgb}{0.000000, 0.000000, 0.000000}
\pgfsetstrokecolor{dialinecolor}
\draw (34.945000\du,42.410000\du)--(34.945000\du,42.810000\du)--(40.855000\du,42.810000\du)--(40.855000\du,42.410000\du)--cycle;
\pgfsetlinewidth{0.100000\du}
\pgfsetdash{}{0pt}
\definecolor{dialinecolor}{rgb}{1.000000, 1.000000, 1.000000}
\pgfsetfillcolor{dialinecolor}
\fill (34.810000\du,29.140000\du)--(34.810000\du,30.540000\du)--(40.720000\du,30.540000\du)--(40.720000\du,29.140000\du)--cycle;
\definecolor{dialinecolor}{rgb}{0.000000, 0.000000, 0.000000}
\pgfsetstrokecolor{dialinecolor}
\draw (34.810000\du,29.140000\du)--(34.810000\du,30.540000\du)--(40.720000\du,30.540000\du)--(40.720000\du,29.140000\du)--cycle;
% setfont left to latex
\definecolor{dialinecolor}{rgb}{0.000000, 0.000000, 0.000000}
\pgfsetstrokecolor{dialinecolor}
\node at (37.765000\du,30.290000\du){Reservation};
\definecolor{dialinecolor}{rgb}{1.000000, 1.000000, 1.000000}
\pgfsetfillcolor{dialinecolor}
\fill (34.810000\du,30.540000\du)--(34.810000\du,34.740000\du)--(40.720000\du,34.740000\du)--(40.720000\du,30.540000\du)--cycle;
\definecolor{dialinecolor}{rgb}{0.000000, 0.000000, 0.000000}
\pgfsetstrokecolor{dialinecolor}
\draw (34.810000\du,30.540000\du)--(34.810000\du,34.740000\du)--(40.720000\du,34.740000\du)--(40.720000\du,30.540000\du)--cycle;
% setfont left to latex
\definecolor{dialinecolor}{rgb}{0.000000, 0.000000, 0.000000}
\pgfsetstrokecolor{dialinecolor}
\node[anchor=west] at (34.960000\du,31.400000\du){+이름};
% setfont left to latex
\definecolor{dialinecolor}{rgb}{0.000000, 0.000000, 0.000000}
\pgfsetstrokecolor{dialinecolor}
\node[anchor=west] at (34.960000\du,32.200000\du){+연락처};
% setfont left to latex
\definecolor{dialinecolor}{rgb}{0.000000, 0.000000, 0.000000}
\pgfsetstrokecolor{dialinecolor}
\node[anchor=west] at (34.960000\du,33.000000\du){+예약시작시간};
% setfont left to latex
\definecolor{dialinecolor}{rgb}{0.000000, 0.000000, 0.000000}
\pgfsetstrokecolor{dialinecolor}
\node[anchor=west] at (34.960000\du,33.800000\du){+예약종료시간};
% setfont left to latex
\definecolor{dialinecolor}{rgb}{0.000000, 0.000000, 0.000000}
\pgfsetstrokecolor{dialinecolor}
\node[anchor=west] at (34.960000\du,34.600000\du){+Node};
\definecolor{dialinecolor}{rgb}{1.000000, 1.000000, 1.000000}
\pgfsetfillcolor{dialinecolor}
\fill (34.810000\du,34.740000\du)--(34.810000\du,35.140000\du)--(40.720000\du,35.140000\du)--(40.720000\du,34.740000\du)--cycle;
\definecolor{dialinecolor}{rgb}{0.000000, 0.000000, 0.000000}
\pgfsetstrokecolor{dialinecolor}
\draw (34.810000\du,34.740000\du)--(34.810000\du,35.140000\du)--(40.720000\du,35.140000\du)--(40.720000\du,34.740000\du)--cycle;
\pgfsetlinewidth{0.100000\du}
\pgfsetdash{}{0pt}
\definecolor{dialinecolor}{rgb}{1.000000, 1.000000, 1.000000}
\pgfsetfillcolor{dialinecolor}
\fill (34.775000\du,21.220000\du)--(34.775000\du,22.620000\du)--(40.685000\du,22.620000\du)--(40.685000\du,21.220000\du)--cycle;
\definecolor{dialinecolor}{rgb}{0.000000, 0.000000, 0.000000}
\pgfsetstrokecolor{dialinecolor}
\draw (34.775000\du,21.220000\du)--(34.775000\du,22.620000\du)--(40.685000\du,22.620000\du)--(40.685000\du,21.220000\du)--cycle;
% setfont left to latex
\definecolor{dialinecolor}{rgb}{0.000000, 0.000000, 0.000000}
\pgfsetstrokecolor{dialinecolor}
\node at (37.730000\du,22.370000\du){Reservation};
\definecolor{dialinecolor}{rgb}{1.000000, 1.000000, 1.000000}
\pgfsetfillcolor{dialinecolor}
\fill (34.775000\du,22.620000\du)--(34.775000\du,26.820000\du)--(40.685000\du,26.820000\du)--(40.685000\du,22.620000\du)--cycle;
\definecolor{dialinecolor}{rgb}{0.000000, 0.000000, 0.000000}
\pgfsetstrokecolor{dialinecolor}
\draw (34.775000\du,22.620000\du)--(34.775000\du,26.820000\du)--(40.685000\du,26.820000\du)--(40.685000\du,22.620000\du)--cycle;
% setfont left to latex
\definecolor{dialinecolor}{rgb}{0.000000, 0.000000, 0.000000}
\pgfsetstrokecolor{dialinecolor}
\node[anchor=west] at (34.925000\du,23.480000\du){+이름};
% setfont left to latex
\definecolor{dialinecolor}{rgb}{0.000000, 0.000000, 0.000000}
\pgfsetstrokecolor{dialinecolor}
\node[anchor=west] at (34.925000\du,24.280000\du){+연락처};
% setfont left to latex
\definecolor{dialinecolor}{rgb}{0.000000, 0.000000, 0.000000}
\pgfsetstrokecolor{dialinecolor}
\node[anchor=west] at (34.925000\du,25.080000\du){+예약시작시간};
% setfont left to latex
\definecolor{dialinecolor}{rgb}{0.000000, 0.000000, 0.000000}
\pgfsetstrokecolor{dialinecolor}
\node[anchor=west] at (34.925000\du,25.880000\du){+예약종료시간};
% setfont left to latex
\definecolor{dialinecolor}{rgb}{0.000000, 0.000000, 0.000000}
\pgfsetstrokecolor{dialinecolor}
\node[anchor=west] at (34.925000\du,26.680000\du){+Node};
\definecolor{dialinecolor}{rgb}{1.000000, 1.000000, 1.000000}
\pgfsetfillcolor{dialinecolor}
\fill (34.775000\du,26.820000\du)--(34.775000\du,27.220000\du)--(40.685000\du,27.220000\du)--(40.685000\du,26.820000\du)--cycle;
\definecolor{dialinecolor}{rgb}{0.000000, 0.000000, 0.000000}
\pgfsetstrokecolor{dialinecolor}
\draw (34.775000\du,26.820000\du)--(34.775000\du,27.220000\du)--(40.685000\du,27.220000\du)--(40.685000\du,26.820000\du)--cycle;
\pgfsetlinewidth{0.100000\du}
\pgfsetbuttcap
\pgfsetdash{}{0pt}
{
\definecolor{dialinecolor}{rgb}{0.000000, 0.000000, 0.000000}
\pgfsetfillcolor{dialinecolor}
% was here!!!
\pgfsetarrowsstart{latex}
\definecolor{dialinecolor}{rgb}{0.000000, 0.000000, 0.000000}
\pgfsetstrokecolor{dialinecolor}
\draw (37.752681\du,21.170282\du)--(37.772319\du,18.529718\du);
}
% setfont left to latex
\pgfsetlinewidth{0.100000\du}
\pgfsetbuttcap
\pgfsetdash{}{0pt}
{
\definecolor{dialinecolor}{rgb}{0.000000, 0.000000, 0.000000}
\pgfsetfillcolor{dialinecolor}
% was here!!!
\pgfsetarrowsstart{latex}
\definecolor{dialinecolor}{rgb}{0.000000, 0.000000, 0.000000}
\pgfsetstrokecolor{dialinecolor}
\draw (37.751523\du,29.090239\du)--(37.743477\du,27.269761\du);
}
% setfont left to latex
\pgfsetlinewidth{0.100000\du}
\pgfsetbuttcap
\pgfsetdash{}{0pt}
{
\definecolor{dialinecolor}{rgb}{0.000000, 0.000000, 0.000000}
\pgfsetfillcolor{dialinecolor}
% was here!!!
\pgfsetarrowsstart{latex}
\definecolor{dialinecolor}{rgb}{0.000000, 0.000000, 0.000000}
\pgfsetstrokecolor{dialinecolor}
\draw (37.846326\du,36.760538\du)--(37.818674\du,35.189462\du);
}
% setfont left to latex
\pgfsetlinewidth{0.100000\du}
\pgfsetdash{}{0pt}
\definecolor{dialinecolor}{rgb}{1.000000, 1.000000, 1.000000}
\pgfsetfillcolor{dialinecolor}
\fill (44.555000\du,12.410000\du)--(44.555000\du,13.810000\du)--(50.465000\du,13.810000\du)--(50.465000\du,12.410000\du)--cycle;
\definecolor{dialinecolor}{rgb}{0.000000, 0.000000, 0.000000}
\pgfsetstrokecolor{dialinecolor}
\draw (44.555000\du,12.410000\du)--(44.555000\du,13.810000\du)--(50.465000\du,13.810000\du)--(50.465000\du,12.410000\du)--cycle;
% setfont left to latex
\definecolor{dialinecolor}{rgb}{0.000000, 0.000000, 0.000000}
\pgfsetstrokecolor{dialinecolor}
\node at (47.510000\du,13.560000\du){Reservation};
\definecolor{dialinecolor}{rgb}{1.000000, 1.000000, 1.000000}
\pgfsetfillcolor{dialinecolor}
\fill (44.555000\du,13.810000\du)--(44.555000\du,18.010000\du)--(50.465000\du,18.010000\du)--(50.465000\du,13.810000\du)--cycle;
\definecolor{dialinecolor}{rgb}{0.000000, 0.000000, 0.000000}
\pgfsetstrokecolor{dialinecolor}
\draw (44.555000\du,13.810000\du)--(44.555000\du,18.010000\du)--(50.465000\du,18.010000\du)--(50.465000\du,13.810000\du)--cycle;
% setfont left to latex
\definecolor{dialinecolor}{rgb}{0.000000, 0.000000, 0.000000}
\pgfsetstrokecolor{dialinecolor}
\node[anchor=west] at (44.705000\du,14.670000\du){+이름};
% setfont left to latex
\definecolor{dialinecolor}{rgb}{0.000000, 0.000000, 0.000000}
\pgfsetstrokecolor{dialinecolor}
\node[anchor=west] at (44.705000\du,15.470000\du){+연락처};
% setfont left to latex
\definecolor{dialinecolor}{rgb}{0.000000, 0.000000, 0.000000}
\pgfsetstrokecolor{dialinecolor}
\node[anchor=west] at (44.705000\du,16.270000\du){+예약시작시간};
% setfont left to latex
\definecolor{dialinecolor}{rgb}{0.000000, 0.000000, 0.000000}
\pgfsetstrokecolor{dialinecolor}
\node[anchor=west] at (44.705000\du,17.070000\du){+예약종료시간};
% setfont left to latex
\definecolor{dialinecolor}{rgb}{0.000000, 0.000000, 0.000000}
\pgfsetstrokecolor{dialinecolor}
\node[anchor=west] at (44.705000\du,17.870000\du){+Node};
\definecolor{dialinecolor}{rgb}{1.000000, 1.000000, 1.000000}
\pgfsetfillcolor{dialinecolor}
\fill (44.555000\du,18.010000\du)--(44.555000\du,18.410000\du)--(50.465000\du,18.410000\du)--(50.465000\du,18.010000\du)--cycle;
\definecolor{dialinecolor}{rgb}{0.000000, 0.000000, 0.000000}
\pgfsetstrokecolor{dialinecolor}
\draw (44.555000\du,18.010000\du)--(44.555000\du,18.410000\du)--(50.465000\du,18.410000\du)--(50.465000\du,18.010000\du)--cycle;
\pgfsetlinewidth{0.100000\du}
\pgfsetdash{}{0pt}
\definecolor{dialinecolor}{rgb}{1.000000, 1.000000, 1.000000}
\pgfsetfillcolor{dialinecolor}
\fill (44.460000\du,36.340000\du)--(44.460000\du,37.740000\du)--(50.370000\du,37.740000\du)--(50.370000\du,36.340000\du)--cycle;
\definecolor{dialinecolor}{rgb}{0.000000, 0.000000, 0.000000}
\pgfsetstrokecolor{dialinecolor}
\draw (44.460000\du,36.340000\du)--(44.460000\du,37.740000\du)--(50.370000\du,37.740000\du)--(50.370000\du,36.340000\du)--cycle;
% setfont left to latex
\definecolor{dialinecolor}{rgb}{0.000000, 0.000000, 0.000000}
\pgfsetstrokecolor{dialinecolor}
\node at (47.415000\du,37.490000\du){Reservation};
\definecolor{dialinecolor}{rgb}{1.000000, 1.000000, 1.000000}
\pgfsetfillcolor{dialinecolor}
\fill (44.460000\du,37.740000\du)--(44.460000\du,41.940000\du)--(50.370000\du,41.940000\du)--(50.370000\du,37.740000\du)--cycle;
\definecolor{dialinecolor}{rgb}{0.000000, 0.000000, 0.000000}
\pgfsetstrokecolor{dialinecolor}
\draw (44.460000\du,37.740000\du)--(44.460000\du,41.940000\du)--(50.370000\du,41.940000\du)--(50.370000\du,37.740000\du)--cycle;
% setfont left to latex
\definecolor{dialinecolor}{rgb}{0.000000, 0.000000, 0.000000}
\pgfsetstrokecolor{dialinecolor}
\node[anchor=west] at (44.610000\du,38.600000\du){+이름};
% setfont left to latex
\definecolor{dialinecolor}{rgb}{0.000000, 0.000000, 0.000000}
\pgfsetstrokecolor{dialinecolor}
\node[anchor=west] at (44.610000\du,39.400000\du){+연락처};
% setfont left to latex
\definecolor{dialinecolor}{rgb}{0.000000, 0.000000, 0.000000}
\pgfsetstrokecolor{dialinecolor}
\node[anchor=west] at (44.610000\du,40.200000\du){+예약시작시간};
% setfont left to latex
\definecolor{dialinecolor}{rgb}{0.000000, 0.000000, 0.000000}
\pgfsetstrokecolor{dialinecolor}
\node[anchor=west] at (44.610000\du,41.000000\du){+예약종료시간};
% setfont left to latex
\definecolor{dialinecolor}{rgb}{0.000000, 0.000000, 0.000000}
\pgfsetstrokecolor{dialinecolor}
\node[anchor=west] at (44.610000\du,41.800000\du){+Node};
\definecolor{dialinecolor}{rgb}{1.000000, 1.000000, 1.000000}
\pgfsetfillcolor{dialinecolor}
\fill (44.460000\du,41.940000\du)--(44.460000\du,42.340000\du)--(50.370000\du,42.340000\du)--(50.370000\du,41.940000\du)--cycle;
\definecolor{dialinecolor}{rgb}{0.000000, 0.000000, 0.000000}
\pgfsetstrokecolor{dialinecolor}
\draw (44.460000\du,41.940000\du)--(44.460000\du,42.340000\du)--(50.370000\du,42.340000\du)--(50.370000\du,41.940000\du)--cycle;
\pgfsetlinewidth{0.100000\du}
\pgfsetdash{}{0pt}
\definecolor{dialinecolor}{rgb}{1.000000, 1.000000, 1.000000}
\pgfsetfillcolor{dialinecolor}
\fill (44.375000\du,28.070000\du)--(44.375000\du,29.470000\du)--(50.285000\du,29.470000\du)--(50.285000\du,28.070000\du)--cycle;
\definecolor{dialinecolor}{rgb}{0.000000, 0.000000, 0.000000}
\pgfsetstrokecolor{dialinecolor}
\draw (44.375000\du,28.070000\du)--(44.375000\du,29.470000\du)--(50.285000\du,29.470000\du)--(50.285000\du,28.070000\du)--cycle;
% setfont left to latex
\definecolor{dialinecolor}{rgb}{0.000000, 0.000000, 0.000000}
\pgfsetstrokecolor{dialinecolor}
\node at (47.330000\du,29.220000\du){Reservation};
\definecolor{dialinecolor}{rgb}{1.000000, 1.000000, 1.000000}
\pgfsetfillcolor{dialinecolor}
\fill (44.375000\du,29.470000\du)--(44.375000\du,33.670000\du)--(50.285000\du,33.670000\du)--(50.285000\du,29.470000\du)--cycle;
\definecolor{dialinecolor}{rgb}{0.000000, 0.000000, 0.000000}
\pgfsetstrokecolor{dialinecolor}
\draw (44.375000\du,29.470000\du)--(44.375000\du,33.670000\du)--(50.285000\du,33.670000\du)--(50.285000\du,29.470000\du)--cycle;
% setfont left to latex
\definecolor{dialinecolor}{rgb}{0.000000, 0.000000, 0.000000}
\pgfsetstrokecolor{dialinecolor}
\node[anchor=west] at (44.525000\du,30.330000\du){+이름};
% setfont left to latex
\definecolor{dialinecolor}{rgb}{0.000000, 0.000000, 0.000000}
\pgfsetstrokecolor{dialinecolor}
\node[anchor=west] at (44.525000\du,31.130000\du){+연락처};
% setfont left to latex
\definecolor{dialinecolor}{rgb}{0.000000, 0.000000, 0.000000}
\pgfsetstrokecolor{dialinecolor}
\node[anchor=west] at (44.525000\du,31.930000\du){+예약시작시간};
% setfont left to latex
\definecolor{dialinecolor}{rgb}{0.000000, 0.000000, 0.000000}
\pgfsetstrokecolor{dialinecolor}
\node[anchor=west] at (44.525000\du,32.730000\du){+예약종료시간};
% setfont left to latex
\definecolor{dialinecolor}{rgb}{0.000000, 0.000000, 0.000000}
\pgfsetstrokecolor{dialinecolor}
\node[anchor=west] at (44.525000\du,33.530000\du){+Node};
\definecolor{dialinecolor}{rgb}{1.000000, 1.000000, 1.000000}
\pgfsetfillcolor{dialinecolor}
\fill (44.375000\du,33.670000\du)--(44.375000\du,34.070000\du)--(50.285000\du,34.070000\du)--(50.285000\du,33.670000\du)--cycle;
\definecolor{dialinecolor}{rgb}{0.000000, 0.000000, 0.000000}
\pgfsetstrokecolor{dialinecolor}
\draw (44.375000\du,33.670000\du)--(44.375000\du,34.070000\du)--(50.285000\du,34.070000\du)--(50.285000\du,33.670000\du)--cycle;
\pgfsetlinewidth{0.100000\du}
\pgfsetdash{}{0pt}
\definecolor{dialinecolor}{rgb}{1.000000, 1.000000, 1.000000}
\pgfsetfillcolor{dialinecolor}
\fill (44.490000\du,20.400000\du)--(44.490000\du,21.800000\du)--(50.400000\du,21.800000\du)--(50.400000\du,20.400000\du)--cycle;
\definecolor{dialinecolor}{rgb}{0.000000, 0.000000, 0.000000}
\pgfsetstrokecolor{dialinecolor}
\draw (44.490000\du,20.400000\du)--(44.490000\du,21.800000\du)--(50.400000\du,21.800000\du)--(50.400000\du,20.400000\du)--cycle;
% setfont left to latex
\definecolor{dialinecolor}{rgb}{0.000000, 0.000000, 0.000000}
\pgfsetstrokecolor{dialinecolor}
\node at (47.445000\du,21.550000\du){Reservation};
\definecolor{dialinecolor}{rgb}{1.000000, 1.000000, 1.000000}
\pgfsetfillcolor{dialinecolor}
\fill (44.490000\du,21.800000\du)--(44.490000\du,26.000000\du)--(50.400000\du,26.000000\du)--(50.400000\du,21.800000\du)--cycle;
\definecolor{dialinecolor}{rgb}{0.000000, 0.000000, 0.000000}
\pgfsetstrokecolor{dialinecolor}
\draw (44.490000\du,21.800000\du)--(44.490000\du,26.000000\du)--(50.400000\du,26.000000\du)--(50.400000\du,21.800000\du)--cycle;
% setfont left to latex
\definecolor{dialinecolor}{rgb}{0.000000, 0.000000, 0.000000}
\pgfsetstrokecolor{dialinecolor}
\node[anchor=west] at (44.640000\du,22.660000\du){+이름};
% setfont left to latex
\definecolor{dialinecolor}{rgb}{0.000000, 0.000000, 0.000000}
\pgfsetstrokecolor{dialinecolor}
\node[anchor=west] at (44.640000\du,23.460000\du){+연락처};
% setfont left to latex
\definecolor{dialinecolor}{rgb}{0.000000, 0.000000, 0.000000}
\pgfsetstrokecolor{dialinecolor}
\node[anchor=west] at (44.640000\du,24.260000\du){+예약시작시간};
% setfont left to latex
\definecolor{dialinecolor}{rgb}{0.000000, 0.000000, 0.000000}
\pgfsetstrokecolor{dialinecolor}
\node[anchor=west] at (44.640000\du,25.060000\du){+예약종료시간};
% setfont left to latex
\definecolor{dialinecolor}{rgb}{0.000000, 0.000000, 0.000000}
\pgfsetstrokecolor{dialinecolor}
\node[anchor=west] at (44.640000\du,25.860000\du){+Node};
\definecolor{dialinecolor}{rgb}{1.000000, 1.000000, 1.000000}
\pgfsetfillcolor{dialinecolor}
\fill (44.490000\du,26.000000\du)--(44.490000\du,26.400000\du)--(50.400000\du,26.400000\du)--(50.400000\du,26.000000\du)--cycle;
\definecolor{dialinecolor}{rgb}{0.000000, 0.000000, 0.000000}
\pgfsetstrokecolor{dialinecolor}
\draw (44.490000\du,26.000000\du)--(44.490000\du,26.400000\du)--(50.400000\du,26.400000\du)--(50.400000\du,26.000000\du)--cycle;
\pgfsetlinewidth{0.100000\du}
\pgfsetbuttcap
\pgfsetdash{}{0pt}
{
\definecolor{dialinecolor}{rgb}{0.000000, 0.000000, 0.000000}
\pgfsetfillcolor{dialinecolor}
% was here!!!
\pgfsetarrowsstart{latex}
\definecolor{dialinecolor}{rgb}{0.000000, 0.000000, 0.000000}
\pgfsetstrokecolor{dialinecolor}
\draw (47.469788\du,20.353032\du)--(47.485212\du,18.456968\du);
}
% setfont left to latex
\pgfsetlinewidth{0.100000\du}
\pgfsetbuttcap
\pgfsetdash{}{0pt}
{
\definecolor{dialinecolor}{rgb}{0.000000, 0.000000, 0.000000}
\pgfsetfillcolor{dialinecolor}
% was here!!!
\pgfsetarrowsstart{latex}
\definecolor{dialinecolor}{rgb}{0.000000, 0.000000, 0.000000}
\pgfsetstrokecolor{dialinecolor}
\draw (47.375722\du,28.020538\du)--(47.399278\du,26.449462\du);
}
% setfont left to latex
\pgfsetlinewidth{0.100000\du}
\pgfsetbuttcap
\pgfsetdash{}{0pt}
{
\definecolor{dialinecolor}{rgb}{0.000000, 0.000000, 0.000000}
\pgfsetfillcolor{dialinecolor}
% was here!!!
\pgfsetarrowsstart{latex}
\definecolor{dialinecolor}{rgb}{0.000000, 0.000000, 0.000000}
\pgfsetstrokecolor{dialinecolor}
\draw (47.383654\du,36.290236\du)--(47.361346\du,34.119764\du);
}
% setfont left to latex
\pgfsetlinewidth{0.100000\du}
\pgfsetdash{}{0pt}
{\pgfsetcornersarced{\pgfpoint{0.500000\du}{0.500000\du}}\definecolor{dialinecolor}{rgb}{1.000000, 1.000000, 1.000000}
\pgfsetfillcolor{dialinecolor}
\fill (17.470000\du,5.400000\du)--(17.470000\du,7.200000\du)--(21.470000\du,7.200000\du)--(21.470000\du,5.400000\du)--cycle;
}{\pgfsetcornersarced{\pgfpoint{0.500000\du}{0.500000\du}}\definecolor{dialinecolor}{rgb}{0.000000, 0.000000, 0.000000}
\pgfsetstrokecolor{dialinecolor}
\draw (17.470000\du,5.400000\du)--(17.470000\du,7.200000\du)--(21.470000\du,7.200000\du)--(21.470000\du,5.400000\du)--cycle;
}% setfont left to latex
\definecolor{dialinecolor}{rgb}{0.000000, 0.000000, 0.000000}
\pgfsetstrokecolor{dialinecolor}
\node at (19.470000\du,6.495000\du){Facility1};
\pgfsetlinewidth{0.100000\du}
\pgfsetdash{}{0pt}
{\pgfsetcornersarced{\pgfpoint{0.500000\du}{0.500000\du}}\definecolor{dialinecolor}{rgb}{1.000000, 1.000000, 1.000000}
\pgfsetfillcolor{dialinecolor}
\fill (27.385000\du,5.430000\du)--(27.385000\du,7.230100\du)--(31.385000\du,7.230100\du)--(31.385000\du,5.430000\du)--cycle;
}{\pgfsetcornersarced{\pgfpoint{0.500000\du}{0.500000\du}}\definecolor{dialinecolor}{rgb}{0.000000, 0.000000, 0.000000}
\pgfsetstrokecolor{dialinecolor}
\draw (27.385000\du,5.430000\du)--(27.385000\du,7.230100\du)--(31.385000\du,7.230100\du)--(31.385000\du,5.430000\du)--cycle;
}% setfont left to latex
\definecolor{dialinecolor}{rgb}{0.000000, 0.000000, 0.000000}
\pgfsetstrokecolor{dialinecolor}
\node at (29.385000\du,6.525000\du){Facility2};
\pgfsetlinewidth{0.100000\du}
\pgfsetdash{}{0pt}
{\pgfsetcornersarced{\pgfpoint{0.500000\du}{0.500000\du}}\definecolor{dialinecolor}{rgb}{1.000000, 1.000000, 1.000000}
\pgfsetfillcolor{dialinecolor}
\fill (35.885000\du,5.630000\du)--(35.885000\du,7.430000\du)--(39.885000\du,7.430000\du)--(39.885000\du,5.630000\du)--cycle;
}{\pgfsetcornersarced{\pgfpoint{0.500000\du}{0.500000\du}}\definecolor{dialinecolor}{rgb}{0.000000, 0.000000, 0.000000}
\pgfsetstrokecolor{dialinecolor}
\draw (35.885000\du,5.630000\du)--(35.885000\du,7.430000\du)--(39.885000\du,7.430000\du)--(39.885000\du,5.630000\du)--cycle;
}% setfont left to latex
\definecolor{dialinecolor}{rgb}{0.000000, 0.000000, 0.000000}
\pgfsetstrokecolor{dialinecolor}
\node at (37.885000\du,6.725000\du){Facility3};
\pgfsetlinewidth{0.100000\du}
\pgfsetdash{}{0pt}
{\pgfsetcornersarced{\pgfpoint{0.500000\du}{0.500000\du}}\definecolor{dialinecolor}{rgb}{1.000000, 1.000000, 1.000000}
\pgfsetfillcolor{dialinecolor}
\fill (45.285000\du,5.630000\du)--(45.285000\du,7.430000\du)--(49.285000\du,7.430000\du)--(49.285000\du,5.630000\du)--cycle;
}{\pgfsetcornersarced{\pgfpoint{0.500000\du}{0.500000\du}}\definecolor{dialinecolor}{rgb}{0.000000, 0.000000, 0.000000}
\pgfsetstrokecolor{dialinecolor}
\draw (45.285000\du,5.630000\du)--(45.285000\du,7.430000\du)--(49.285000\du,7.430000\du)--(49.285000\du,5.630000\du)--cycle;
}% setfont left to latex
\definecolor{dialinecolor}{rgb}{0.000000, 0.000000, 0.000000}
\pgfsetstrokecolor{dialinecolor}
\node at (47.285000\du,6.725000\du){Facility4};
% setfont left to latex
\definecolor{dialinecolor}{rgb}{0.000000, 0.000000, 0.000000}
\pgfsetstrokecolor{dialinecolor}
\node[anchor=west] at (19.470000\du,6.300000\du){};
\pgfsetlinewidth{0.100000\du}
\pgfsetbuttcap
\pgfsetdash{}{0pt}
{
\definecolor{dialinecolor}{rgb}{0.000000, 0.000000, 0.000000}
\pgfsetfillcolor{dialinecolor}
% was here!!!
\pgfsetarrowsstart{latex}
\definecolor{dialinecolor}{rgb}{0.000000, 0.000000, 0.000000}
\pgfsetstrokecolor{dialinecolor}
\draw (27.335916\du,6.323840\du)--(21.519084\du,6.306210\du);
}
% setfont left to latex
\pgfsetlinewidth{0.100000\du}
\pgfsetbuttcap
\pgfsetdash{}{0pt}
{
\definecolor{dialinecolor}{rgb}{0.000000, 0.000000, 0.000000}
\pgfsetfillcolor{dialinecolor}
% was here!!!
\pgfsetarrowsstart{latex}
\definecolor{dialinecolor}{rgb}{0.000000, 0.000000, 0.000000}
\pgfsetstrokecolor{dialinecolor}
\draw (35.838857\du,6.481868\du)--(31.431143\du,6.378182\du);
}
% setfont left to latex
\pgfsetlinewidth{0.100000\du}
\pgfsetbuttcap
\pgfsetdash{}{0pt}
{
\definecolor{dialinecolor}{rgb}{0.000000, 0.000000, 0.000000}
\pgfsetfillcolor{dialinecolor}
% was here!!!
\pgfsetarrowsstart{latex}
\definecolor{dialinecolor}{rgb}{0.000000, 0.000000, 0.000000}
\pgfsetstrokecolor{dialinecolor}
\draw (45.235061\du,6.530000\du)--(39.934939\du,6.530000\du);
}
% setfont left to latex
\pgfsetlinewidth{0.100000\du}
\pgfsetbuttcap
\pgfsetdash{}{0pt}
{
\definecolor{dialinecolor}{rgb}{0.000000, 0.000000, 0.000000}
\pgfsetfillcolor{dialinecolor}
% was here!!!
\pgfsetarrowsstart{latex}
\definecolor{dialinecolor}{rgb}{0.000000, 0.000000, 0.000000}
\pgfsetstrokecolor{dialinecolor}
\draw (19.692747\du,12.350745\du)--(19.504841\du,7.246436\du);
}
% setfont left to latex
\pgfsetlinewidth{0.100000\du}
\pgfsetbuttcap
\pgfsetdash{}{0pt}
{
\definecolor{dialinecolor}{rgb}{0.000000, 0.000000, 0.000000}
\pgfsetfillcolor{dialinecolor}
% was here!!!
\pgfsetarrowsstart{latex}
\definecolor{dialinecolor}{rgb}{0.000000, 0.000000, 0.000000}
\pgfsetstrokecolor{dialinecolor}
\draw (29.216232\du,12.299766\du)--(29.358152\du,7.279721\du);
}
% setfont left to latex
\pgfsetlinewidth{0.100000\du}
\pgfsetbuttcap
\pgfsetdash{}{0pt}
{
\definecolor{dialinecolor}{rgb}{0.000000, 0.000000, 0.000000}
\pgfsetfillcolor{dialinecolor}
% was here!!!
\pgfsetarrowsstart{latex}
\definecolor{dialinecolor}{rgb}{0.000000, 0.000000, 0.000000}
\pgfsetstrokecolor{dialinecolor}
\draw (37.825630\du,12.434028\du)--(37.875453\du,7.479408\du);
}
% setfont left to latex
\pgfsetlinewidth{0.100000\du}
\pgfsetbuttcap
\pgfsetdash{}{0pt}
{
\definecolor{dialinecolor}{rgb}{0.000000, 0.000000, 0.000000}
\pgfsetfillcolor{dialinecolor}
% was here!!!
\pgfsetarrowsstart{latex}
\definecolor{dialinecolor}{rgb}{0.000000, 0.000000, 0.000000}
\pgfsetstrokecolor{dialinecolor}
\draw (47.432739\du,12.360752\du)--(47.309074\du,7.480112\du);
}
% setfont left to latex
\end{tikzpicture}

\paragraph{테이블 표현 양식의 예}
예약상황의 빠른 파악을 위해 다음과 같은 형식으로 시간 혹은 일자로 스케일을 바꾸어 가며 표현할 수 있게 한다. 또한 테이블의 칸을 클릭하였을 때 빈칸일 경우는 예약, 예약된 칸일 경우 삭제를 할 수 있게 한다.\\
\begin{tabular}{|c|c|c|c|c|c|c|c|c|}
\hline
시설명& 1&2&3&4&5&6&7&8 날짜 혹은 시간 ......\\
\hline
청실& \multicolumn{3}{c|}{박승원} & \multicolumn{5}{c|}{김말년}\\
\hline
홍실& \multicolumn{4}{c|}{손오공}& \multicolumn{2}{c|}{저팔계}&\multicolumn{2}{c|}{사오정}\\
\hline
금실& \multicolumn{4}{c|}{}& \multicolumn{2}{c|}{저팔계}&\multicolumn{2}{c|}{사오정}\\
\hline
은실& \multicolumn{4}{c|}{손오공}& \multicolumn{2}{c|}{저팔계}&\multicolumn{2}{c|}{}\\
\hline

\end{tabular}
\paragraph{시설 설정 파일의 예}facility.txt\\
청실\\홍실\\금실\\은실\\EOF

\subsection{마무리}
서버와 클라이언트는 어느 시스템에서든 호환 가능하도록 호환성을 중심으로 작성하고, 프론트 엔드는 에러에 어느 정도 대처할 수 있는 사용자의 예기치 않은 입력에 대응할 수 있는 프로그램을 추구한다.

프론트 엔드는 각각의 운영체제의 라이브러리에 맞추어 개발할 여지를 남겨둔다.
\section{과제 수행 방법}
매우 짧은 개발 일정과 자료구조 연습이라는 목적을 기준으로, 개발의 우선순위를 두기로 하였다.
우선은 서버와 커맨드 라인 클라이언트를 작성하고, 이후에 프론트 엔드를 하기로 하였다. 
TCPIP 통신 모듈은 기존에 작성하였던 것을 이용하기로 하였다. 
\chapter{수행 일정}
\begin{tabular}{|c|c|}
	\hline
세부 개발내용&주별 세부 추진 일정\\
\hline
자료수집 및 설계 계획 수립&$\sim$ 10/16\\
요구사항 검토&10.17\\
구조도 및 모듈 설계&10.18\\
프로그램 설계&10.19$\sim$ 10.21\\
프로그램 코딩&10.21$\sim$ 10.25\\
성능확인 및 오류 수정&10.26$\sim$ 10.27\\
최종 점검 및 개선 사항 보완&10.28\\
\hline
\end{tabular}
\end{document}
